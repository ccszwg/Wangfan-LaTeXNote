\section{超链接}
\subsection{选项配置}
\textcolor[rgb]{0.50,0.00,0.50}{ 颜色的应用:} \newline 1.
红色(Red)用来作为内部的链接 \newline 2.
红紫色(Magenta)用来作为web或者email地址的链接。 \newline 3.
蓝色(有时候是绿色,还有少数是红色)用来标示强调的词或者说明性的词。
\newline
\textcolor[rgb]{0.50,0.00,0.50}{书签(Bookmarks) }\newline 1.
Options:选择hyperref的四个位置: \newline (a) 全局: $\backslash
$documentclass[. . . ] \newline (b) 用宏包: $\backslash
$usepackage[. . . ] \newline (c) 配置文件: 用$\backslash
$hypersetup配置hyperref.cfg \newline (d) 在宏包之后: $\backslash
$hypersetup{\{}. . . {\}} \newline 2. 影响书签的选项 \newline (a)
bookmarks:使得书签有效(缺省为true).如果宏包已经装载,那么这个选项不能应用。
\newline
(b) bookmarksnumbered: 把节的标号放入标签(缺省:false)。 \newline (c)
bookmarksopen: 打开书、签树(缺省: false). \newline (d)
bookmarksopenlevel:书签打开的层数(缺省: maxdimen). \newline

\textcolor[rgb]{0.50,0.00,0.50}{ hyperref的设置} \newline
1.应用如下设置超链接的颜色是相同的 \newline $\backslash
$documentclass[11pt]{\{}article{\}} \newline $\backslash
$usepackage{\{}color{\}} \newline $\backslash $ifx$\backslash
$pdfoutput$\backslash $undefined \newline $\backslash
$usepackage[dvips]{\{}graphicx{\}} \newline $\backslash
$usepackage[dvipdf, \newline pdfstartview=FitH, CJKbookmarks=true,
bookmarksnumbered=true, \newline bookmarksopen=true,
colorlinks=false, colorlinks=black, \newline pdfborder=100,
citecolor=black ]{\{}hyperref{\}} \newline $\backslash
$AtBeginDvi{\{}$\backslash $special{\{}pdf:tounicode
GBK-EUC-UCS2{\}}{\}} {\%} GBK -> Unicode \newline $\backslash $else
\newline $\backslash $usepackage[dvipdf, \newline pdfstartview=FitH,
CJKbookmarks=true, bookmarksnumbered=true, \newline
bookmarksopen=true, colorlinks=true, citecolor=black
]{\{}hyperref{\}}
\newline
$\backslash $usepackage[pdftex]{\{}graphicx{\}} \newline $\backslash
$fi \newline 2.应用如下设置超链接的颜色不同的 \newline $\backslash
$ifx$\backslash $pdfoutput$\backslash $undefined \newline
$\backslash $usepackage[CJKbookmarks=true,dvipdfm, \newline
pdfstartview=FitH,{\%}开始时候页面的大小,是正常的页面,即100{\%},注释掉的话页面大小为50{\%}
\newline
]{\{}hyperref{\}} $\backslash $else \newline $\backslash
$usepackage[CJKbookmarks=true,pdftex, \newline
pdfstartview=FitH,{\%}开始时候页面的大小,是正常的页面,即100{\%},注释掉的话页面大小为50{\%}
]{\{}hyperref{\}}

$\backslash $fi \newline
这里仅仅说明了hyperref的设置,可以照旧应用其他宏包(如CJK)。但是注意hyperref应该放在所有宏包的最后。
\newline

hyperref 的选项设置如\ref{hyperref_config}所示。

\begin{center}
\rowcolors{1}{lightgray}{}
\begin{longtable}[H]{p{3.2cm}p{3cm}p{3cm}p{3cm}}
\caption{hyperref 参数选项设置} \label{hyperref_config} \\
\toprule \multicolumn{1}{c}{\textbf{选项}} &
\multicolumn{1}{c}{\textbf{作用 }} &
\multicolumn{1}{l}{\textbf{参数 }}  &
\multicolumn{1}{c}{\textbf{说明 }}\\ \midrule
\endfirsthead

\multicolumn{4}{c}%
{{\centering \kai\thetable{}hyperref 参数选项设置-~- 接上页}} \\
\toprule \multicolumn{1}{c}{\textbf{选项}} &
\multicolumn{1}{c}{\textbf{作用 }} &
\multicolumn{1}{l}{\textbf{参数 }}  &
\multicolumn{1}{c}{\textbf{说明 }}\\ \midrule
\endhead
\multicolumn{3}{r}{{\kai 接下页}} \\ \bottomrule
\endfoot
\bottomrule
\endlastfoot

a4paper & 使用A4纸张 && \\
a5paper & 使用A5纸张 && \\
anchorcolor &链接锚文本颜色 & black &默认值\\
b5paper &使用B5纸张&& \\
backref &反向引用&false &默认值\\
baseurl &基本URL &empty &默认值\\
bookmarks &生成书签&true &默认值\\
bookmarksnumbered &书签中章节编号&true &默认值\\
bookmarksopen &书签目录展开&true &默认值\\
bookmarksopenlevel &书签目录层次&
\verb|\maxdimen| &默认值,最高1或2或3... 可选值\\

%%%%%%%%%%%%%%%%%%%%%%%%%%%
bookmarkstype
&
书签目录类型
&
toc\par
lof\par
lot
&
章节目录,默认\par
图形目录 \par
表格目录
\\
%%%%%%%%%%%%%%%%%%%%%%%%%%%
breaklinks  &允许链接断行 &false& 默认值\\
citebordercolor &引用标记边框颜色&{0 1 0}& 默认值\\
citecolor &引用标记颜色&green& 默认值\\
colorlinks &彩色链接&true &默认值\\
CJKbookmarks &中文书签&true &默认值\\
debug & log诊断信息打印&false& 默认值\\
draft & 超文本选项失效&false& 默认值\\
dvipdf &使用dvipdf驱动&&\\
dvipdfm &使用dvipdf驱动&&\\
dvips &使用dvips驱动&&\\
dvipsone &使用dvipsone驱动&&\\
dviwindo &使用dviwindo驱动&&\\
encap &设置超索引符号&&\\
executivepaper &7.25in×10.5in纸&&\\
extension &设置文件扩展名
&
dvi\par
ps、pdf
&
默认值\par
... 可选值
\\
filebordercolor &文件链接边框颜色&{0 .5 .5}& 默认值\\
filecolor &文件链接颜色&cyan &默认值\\
final& 超文本选项生效&true &默认值\\
frenchlinks &链接字体为小型大写&false& 默认值\\
hyperfigures &图形链接&false &默认值\\
hyperindex &索引链接&true &默认\\
hypertex &hypertex驱动&&\\
hypertexnames &用推测名称链接&true &默认值\\
implicit &内部定义&true &默认值\\

latex2html &latex2html驱动&&\\
legalpaper &8.5in×14in纸张&&\\
letterpaper &8.5in×11in纸张&&\\
linkbordercolor &内部链边框接颜色&{1 0 0} &默认值\\
linkcolor &内部链接颜色&red &默认值\\
linktocpage &目录页码链接&false& 默认值\\
menubordercolor &菜单链接框颜色&{1 0 0} &默认值\\
menucolor& 菜单链接颜色&red &默认值\\
naturalnames &使用编译名&false &默认值\\
nesting & 允许链接嵌套&false& 默认值\\
pageanchor &每页增设页锚&true& 默认值\\
pagebackref &反向引用页码&false &默认值\\
pagebordercolor &页链接框颜色&{1 1 0} &默认值\\
pagecolor &页链接颜色&red &默认值\\

pdfauthor &作者&&\\

%%%%%%%%%%%%%%%%%%%%%%%%%%%%%
pdfborder
&
链接边框
&
{0 0 0} \par
{0 0 1}
&
默认值,无框 \par
可选值,有框
\\
%%%%%%%%%%%%%%%%%%%%%%%%%%%%%%%%%%
pdfborderstyle
&
连接边框样式
&
{/S/U} \par
{/S/D/D[3 2]/W 1}
&
下划线\par
虚线框\\

pdfcenterwindow &在屏幕上居中窗口&true& 默认值\\
pdfcreator& 应用程序(需用命令\verb|\hypersetup|设置)&
LaTeX with hy-perref
package &默认值\\

pdfdirection &
方向设定
&
L2R\par
R2L
&
由左至右\par
由右至左\\
pdfdisplaydoctitle &显示文件标题&true &默认值\\

%%%%%%%%%%%%%%%%%%%%%%%%%%%%%%%%%%%%%%%%%%%%
pdfduplex &双面打印
&
Simplex \par
DuplexFlipShort-Edge \par
DuplexFlipLong-Edge
&
单面 \par
双面短边装订 \par
双面长边装订\\

%%%%%%%%%%%%%%%%%%%%%%%%%%%%%%%%%%%%
pdfescapeform &容错&false &默认值\\


pdffitwindow &调整窗口&false &默认值\\
pdfhighlight &点击链接时显示
&
/I \par
/N \par
/O \par
/P
&
翻转,默认值 \par
外观不变 \par
出现黑色边框 \par
出现黑色半框\\
%%%%%%%%%%%%%%%%%%%%%%%%%%%%%%%%%%%%

pdfkeywords &关键字&&\\
pdflang &PDF标识符&&\\
pdfmenubar &显示菜单栏&true &默认值\\
pdfnewwindow &生成新窗口&false &默认值\\
pdfnumcopies &打印份数 1 或 2 或 3...&&\\

%%%%%%%%%%%%%%%%%%%%%%%%%%%%%%%%%%%%%%%%%
pdfnonfullscreen-\par
pagemode
&
全屏显示样式
&
UseNone \par
UseOutlines \par
UseThumbs \par
FullScreen
&
无书签缩略图 \par
附书签 \par
附缩略图 \par
无书签缩略图
\\
%%%%%%%%%%%%%%%%%%%%%%%%%%%%%%%%%%%%%%%%%
pdfpagelayout
&
页面布局
&
TwoColumnLeft \par
SinglePage \par
OneColumn \par
TwoColumnRight
&
对开,默认值 \par
单页 \par
连续 \par
连续-对开
\\
%%%%%%%%%%%%%%%%%%%%%%%%%%%%%%%%%%%%%%%%%
pdfpagemode &
文件打开模式
&
UseNone 默认\par
UseThumbs\par
UseOutlines\par
FullScreen
&
无书签和缩略图\par
显示缩略图\par
显示书签\par
全屏显示
\\
pdfpagelabels &
底部页码样式:\par
“v(第 5/15 页)”\par
或“第 5/15 页”
&true& 默认值\\

pdfpagescrop & 设置裁切尺寸&例如:\{53 486 389 754\}&\\

pdfpagetransition &
页面过渡样式\par
参数后可加子参数:\par
/Dm、/Bi、/M、/H
/V、/I、/O\par
(需全屏显示模式)
&
Dissolve \par
Wipe \par
Split \par
Glitter \par
Blinds \par
Box
&
马赛克散开 \par
下拉帘幕 \par
上下拉帘幕 \par
溶化 \par
百叶窗翻转 \par
渐缩框
\\

pdfpicktraybypdfsize  & 纸张自动选择&true& 默认值\\
pdfprintarea & 打印范围&参数与pdfviewarea相同&\\
pdfprintclip & &参数与pdfviewarea相同&\\
pdfprintpagerange & 设置打印页码范围{n n}&&\\

pdfprintscaling &
打印放大率
&
AppDefault  \par
None
&
默认\par
无
\\

pdfproducer &PDF加工程序需用\verb|\hypersetup| 命令设置&&\\
pdfstartpage &打开到页码& 1 &默认值\\

%%%%%%%%%%%%%%%%%%%%%%%%%%%%%%%%%%%%%%%%%
pdfstartview &
PDF文件初始视图&
/Fit \par
FitH \par
FitV \par
FitR \par
FitB \par
FitBH \par
FitBV \par
XYZ \par
&
默认值 \par
页宽适合窗口 \par
页高适合窗口 \par
适合窗口对角线\par
版面适合窗口\par
版宽适合窗口\par
版高适合窗口\par
自定放大率\par
\\
%%%%%%%%%%%%%%%%%%%%%%%%%%%%%%%%%%%%%%%%%
pdfsubject &文件主题 &&\\
pdftex &pdflatex驱动 &&\\
pdftitle &文件标题 &&\\
pdftoolbar &显示工具栏&true& 默认值\\
pdfview &链接默认视图&参数与pdfstartview相同&\\

%%%%%%%%%%%%%%%%%%%%%%%%%%%%%%%%%%%%%%%%%
pdfviewarea &显示区域 &&\\
MediaBox & 媒体框 &&\\
CropBox &裁切框 &&\\
BleedBox &出血框 &&\\
TrimBox &修剪框 &&\\
ArtBox &作品框 &&\\

%%%%%%%%%%%%%%%%%%%%%%%%%%%%%%%%%%
pdfviewclip &剪贴区域&参数与pdfviewarea相同&\\
pdfwindowui &显示窗口控件 &true& 默认值\\
plainpages &页锚编号&true &默认值\\

ps2pdf &ps2pdf驱动&&\\
raiselinks &抬高链接&false &默认值\\
runbordercolor &run链接边框颜色&\{0 .7 .7\} &默认值\\
setpagesize &用命令设置页面尺寸&true &默认值\\

tex4ht& TeX4ht驱动&&\\
textures &Textures 驱动&&\\
unicode &Unicode 编码书签&false& 默认值\\
urlbordercolor &URL 链接边框颜色&\{0 1 1\} &默认值\\
urlcolor &网页与电邮链接颜色&magenta &默认值\\
verbose &附加信息&false &默认值\\
vtex &使用VTeX驱动&&\\
vtexpdfmark &vtexpdfmark驱动&&\\
xetex &使用XeTeX驱动&&\\
\end{longtable}
\end{center}
\subsection{文本超链接}

设置文本超链接,从文本的一处链接到另一处。 \newline 例如: \newline
$\backslash $hypertarget{\{}bilevel{\}}{\{}bilevel programming
problem{\}}设置锚点 \newline $\backslash
$hyperlink{\{}bilevel{\}}{\{}BLP{\}} 设置链接 \newline
\newline
\textcolor[rgb]{0.50,0.00,0.50}{效果为:} \newline bilevel
programming problem \newline BLP \newline 则点BLP可以链接到bilevel
programming problem \newline

\subsection{EMAIL 超链接}

链接到email \newline 例如: \newline $\backslash
$href{\{}mailto:username@whu.edu.cn{\}}{\{}username@whu.edu.cn{\}}
\newline \textcolor[rgb]{0.50,0.00,0.50}{效果为:} \newline
username@whu.edu.cn \newline
username@whu.edu.cn为显示的内容可以修改为你需要的内容。 \newline

\subsection{网址超链接}
网址的超链接(url,href等的应用) \newline 例如: \newline $\backslash
$url{\{}http://www.whu.edu.cn/{\}}链接到http://www.whu.edu.cn/
\newline
效果为: \newline http://www.whu.edu.cn/ \newline $\backslash
$nolinkurl{\{}http://www.whu.edu.cn/{\}}是无链接输入网址的形式。
\newline
$\backslash
$href{\{}http://www.whu.edu.cn/{\}}{\{}武汉大学{\}}链接到http://www.whu.edu.cn/
\newline
\textcolor[rgb]{0.50,0.00,0.50}{效果为:} \newline 武汉大学 \newline
用$\backslash $href也可以链接硬盘上的文件,例如: \newline
$\backslash $href{\{}C:/paper/example1.pdf{\}}{\{}example1.pdf{\}}
\newline 效果为: \newline example1.pdf \newline
在Dvi文件下可以正常打开文件example1.pdf,但是需要在acrobat中设置才能正常用acrobat来打开文件example1.pdf。而且该命令还支持相对路径,但是路径中不能用中文。
\newline
\subsection{直接打开链接文件}
只要超级链接的地址前加上run:就可以了。例如,用hyperref宏包的href命令创建一个链接打开一个文件some.file就可以用$\backslash$href\{run:path/to/some.file\}\{some
link\}。
这样的话,我们就可以直接在PDF中点击链接打开文件,而不必先切换界面再去点文件了。其实run跟http、ftp等一样,都是协议。

\subsection{图表超链接}
例如: \newline $\backslash $begin{\{}figure{\}} \newline
{\{}$\backslash
$includegraphics[width=2.53in,height=1.75in]{\{}figurename.eps{\}}{\}}
\newline
$\backslash $caption{\{}caption of figure{\}} $\backslash
$label{\{}label of figure{\}} \newline $\backslash
$end{\{}figure{\}} \newline $\backslash $begin{\{}table{\}} \newline
..... $\backslash $caption{\{}caption of table{\}} $\backslash
$label{\{}label of table{\}} \newline $\backslash $end{\{}table{\}}
\newline 然后用ref{\{}label of figure(table){\}}
就可以得到图或表引用的超链接,但这仅仅将编号的数字变成超链接,即图1,表2,如将相应的文字也超链接,即:图1,表2,则可以利用hyperref
包中的$\backslash $autoref功能。只要将引用图的$\backslash
$ref{\{}label{\}}换成$\backslash
$autoref{\{}label{\}}即可。但有个问题就是,这样处理的结果是英文的,即``Figure
1'',或是"Table
2"。要想超链接的文字变成``图1''或者"表2",可以预先做如下的重定义:
\newline $\backslash $renewcommand$\backslash
$figureautorefname{\{}图{\}}. $\backslash $renewcommand$\backslash
$tableautorefname{\{}表{\}}. \newline
对于公式、章节等要达到同等的效果可以类似考虑,不再一一说明。
\newline 注意:如果在图或者表格环境中没有caption或者它与$\backslash
$label分开那么引用时没有图表编 \newline 号。即当用$\backslash
$autoref{\{}label{\}}时仅有Table(Figure)。 \newline

\subsection{公式超链接}
例如: \newline $\backslash $begin{\{}equation{\}}$\backslash
$label{\{}equation{\}}
\newline
x+y=z \newline $\backslash $end{\{}equation{\}} \newline
用ref{\{}label of equation{\}}
可以将公式的引用作为超链接。对于能够产生公式编号的数学环境命令类似考虑,不再一一举例说明。
\newline

\subsection{文献目录超链接}
只要用了宏包hyperref就可以产生他们的超链接。对于文献的引用方法没有改变,但是hyperref宏包不能与cite宏包同时应用,
而且也不能使得文献压缩引用。 \newline

\subsection{定理引理推论中文环境超链接}
例如: \newline 在前面定义$\backslash
$newtheorem{\{}Theorem{\}}{\{}Theorem{\}}[section],由代码 \newline
$\backslash $begin{\{}Theorem{\}}$\backslash $label{\{}theorem1{\}}
\newline 这是一个定理的例子 \newline $\backslash $end{\{}Theorem{\}}
\newline 4 \newline 然后用$\backslash
$ref{\{}theorem1{\}}就可以得到该定理的超链接。其它的类似考虑。对于enumerate,itemize等环境也可以得到引用的超链接。
\newline
\subsection{章节的引用}
例如: \newline $\backslash $section{\{}section's
name{\}}$\backslash $label{\{}label of section's name{\}} \newline
用ref{\{}label of section's name{\}}
可以将节的引用作为超链接,对于部分、章以及子节等类似考虑。 \newline
\subsection{注释超链接}
把$\backslash $ref替换为$\backslash $ref*可以注释掉超链接的形式。
\newline

\subsection{PDF 属性设置}
以下命令放在文档中。
\begin{lstlisting}
  \hypersetup{
  pdfkeywords={latex,pgf,asy,beamer}, %`关键词`
  pdfsubject={latex}, %`主题`
  pdfauthor={`王凡`}, %`作者`
  pdftitle={xelatex `笔记`}, %`标题`
  pdfcreator={texlive2011}}
\end{lstlisting}
