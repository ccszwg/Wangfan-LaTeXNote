\section{字体形式}
\subsection{复制中文,符号。 ccmap Times 宏包}
\color{red}
在 PDFLaTeX 中,调用 ccmap 宏包可使中文字体可以复制,不调用则复制的中文会乱码。

注意:有些字体从PDF中复制过去的代码其中单,双引号等英文中的引号形式复制到WORD中会变成中文格式的引号,注意替换。如~\verb|'|~会变成 ‘,~\verb|"|~  会变成 “ 。在 PDFLATEX 中 引用 Times 宏包可使引号复制正常,而在 \XeLaTeX 中,加入 Times 宏包则会使引号复制不正常。
\normalcolor

\subsection{PDFLaTeX 中的字体设置 zhmCJK 宏包}
相比 CJK 宏包,zhmCJK 宏包有以下优点,zhmCJK 不在 TEXlive 中,需单独下载安装:
\begin{itemize}
  \item 编码可自定义为 GBK 或 UTF-8 ,CJK 的 GBK 编码生成的 PDF 目录标签需额外加上 gbk2uni 程序,否则会乱码
  \item CJK 字体可调用的中文字体有限制,生成中文字体繁琐,zhmCJK 可直接调用系统的各种字体,更加便捷.
  \item zhmCJK 语法命令更多,可控制字体的方式更多。
\end{itemize}
\begin{lstlisting}[language={[LaTeX]TeX}]
%% 编码可为 UTF8( 默认 ),BG5,GBK
\usepackage[encoding=GBK]{zhmCJK}  
%% 全局设置
\setCJKmainfont[BoldFont=simhei.ttf,ItalicFont=simkai.ttf]{simsun.ttc}
\setCJKsansfont[AutoFakeBold=0]{simhei.ttf}
\setCJKmonofont[AutoFakeSlant]{simfang.ttf}
\setCJKfamilyfont{xinwei}{STXINWEI.TTF}

%% 以下可用系统的其他字体 { 调用名称 }{ 系统字体文件 }
\setCJKfamilyfont{song}{simsun.ttc}
\setCJKfamilyfont{hei}{simhei.ttf}
\setCJKfamilyfont{kai}{simkai.ttf}
\setCJKfamilyfont{li}{simli.ttf}
\setCJKfamilyfont{you}{simyou.ttf}
%% 设置后可用 \CJKfamily{song} 类似命令转换字体

\newcommand{\song}{\CJKfamily{song}}
\newcommand{\hei}{\CJKfamily{hei}}
\newcommand{\kai}{\CJKfamily{kai}}
\newcommand{\fs}{\CJKfamily{fs}}
\newcommand{\li}{\CJKfamily{li}}
\newcommand{\you}{\CJKfamily{you}}
\newcommand{\wei}{\CJKfamily{xinwei}}
%% 设置后可用 \song 等命令简化字体调用命令
\end{lstlisting}
\subsection{XeLaTeX 中的字体设置}
主要采用的是 XeCJK 宏包。其中用 CTEX 系列宏包可自动引用 XeCJK 宏包。

\textbf{设置中文字体}
\begin{lstlisting}[language={[LaTeX]TeX}]
\XeTeXlinebreaklocale "zh"
\XeTeXlinebreakskip = 0pt plus 1pt minus 0.1pt
%% 以上两行使中文能自动换行

\newcommand{\song}{\songti}
\newcommand{\fs}{\fangsong}
\newcommand{\hei}{\heiti}
\newcommand{\kai}{\kaishu}
\newcommand{\li}{\lishu}
\newcommand{\you}{\youyuan}
%% 以上用于简化CTeX 宏包的字体命令。
\setCJKmainfont[<fontfeatures>]{<fontname>}
\setCJKsansfont[<fontfeatures>]{<fontname>}
\setCJKmonofont[<fontfeatures>]{<fontname>}
\setCJKfamilyfont{<familyname>}[<fontfeatures>]{<fontname>}
\setCJKmainfont[BoldFont={AdobeHeitiStd},ItalicFont={AdobeKaitiStd}]
{AdobeSongStd} %同时定义粗体斜体字体的定义
\setCJKmainfont{SimSun}   % 设置默认中文字体
\setCJKmonofont{SimSun}   % 设置等宽字体

\end{lstlisting}


\textbf{设置英文字体}

\index{命令!\verb$\setmainfont$}
\index{命令!\verb$\setsansfont$}
\index{命令!\verb$\setmonofont$}
\begin{lstlisting}[language={[LaTeX]TeX}]
%\setmainfont{Times New Roman}
%设置默认英文字体。
\setmainfont{TeX Gyre Pagella}
\setsansfont{TeX Gyre Adventor}
\setmonofont{TeX Gyre Cursor}
\end{lstlisting}
\subsection{数字形式}

\index{命令!\verb$\pagenumbering$}

可用各种形式对页码的编号进行指定,共 5 种。
\begin{lstlisting}[language={[LaTeX]TeX}]
\pagenumbering{Roman}//`页码格式`
\end{lstlisting}
\begin{enumerate}
  \item arabic 阿拉伯数字 1 2
  \item roman 小写罗马 i ii
  \item Roman 大写罗马 I II
  \item alph 小写拉丁 a b c d
  \item Alph 大写拉丁 A B C D
\end{enumerate}
\subsection{英文字体}
在新字体选择方案中,每种字体有 5 种属性,编码、族、形状、系列、尺寸,编码一般不涉及:\\
\begin{enumerate}
  \item 族(family):概观样式, \textcolor[rgb]{0.50,0.00,1.00}{rmfamily sffamily ttfamily} 罗马,无衬线,打字机三种样式
  \item 形状(shape): 倾斜和高矮, \textcolor[rgb]{0.50,0.00,1.00}{upshape itshape slshape scshape} 直立,意大利斜体,slanted 斜体,小体大写字体
  \item 系列(serial): 宽度和黑度,\textcolor[rgb]{0.50,0.00,1.00}{mdseries bfseries} 中等,粗体
  \item 尺寸:从小到大为 \textcolor[rgb]{0.50,0.00,1.00}{tiny scriptsize footnotesize small normalsize large Large LARGE huge Huge}
\end{enumerate}


\subsection{颜色}

注意:box 内一般为单行文本,如要多行文本,要再加上 parbox 命令。\\

\index{命令!\verb$\color$}
\index{命令!\verb$\colorbox$}
\index{命令!\verb$\fcolorbox$}

\begin{center}
\color{red} red \quad \color{green} green \quad\color{blue} blue\\
\color{yellow} yellow \quad  \color{cyan} cyan \quad\color{magenta} magenta\\
\color{black} black \quad \color{black} white \quad\color{orange} orange\\
\color{violet} violet \quad \color{purple} purle \quad\color{brown} brown\\
\color{pink} pink \quad \color{olive} olive \quad\color{darkgray} darkgray\\
\color{gray} gray \quad\color{lightgray} lightgray\\
\end{center}

\normalcolor
\begin{lstlisting}[language={[LaTeX]TeX}]
\usepackage{xcolor}
\color{`色彩`} \textcolor{`色彩`}{`文本`} \pagecolor{`色彩`}{`文本`}
\colorbox{`文本色彩`}{`文本`}
\fcolorbox{`边框色彩`}{`背景色彩`}{`文本`} \normalcolor
//`前言未尾所使用的颜色,一般为黑`
\end{lstlisting}

