\section{代码抄录,加入外部文件}
\subsection{listings 宏包}

\index{宏包!listings}
\index{环境!lstlisting}
代码的粘贴复制主要是采用 listings
的宏包,具体用法参考宏包手册,以下是在 latex 中的用法样例:

\begin{lstlisting}[language={[LaTeX]TeX}]

`以下放在导言区` \lstset{numbers=left,numbersep=5pt,
numberstyle=\tiny\rmfamily\itshape\mdseries,
%`行号显示在代码左边,行号字体为tiny,斜体,中等加粗,行号距程序 5pt,`
basicstyle=\small\kai\upshape\mdseries,
%`代码的字号楷体,small,中等加粗`
backgroundcolor=\color{lightgray},
%`背景为灰色`
keywordstyle=\color{blue!70}\bfseries,
%`关键字高亮加粗`
stringstyle=\ttfamily\itshape,
%`打字机字体`
showstringspaces=false,
%`不显示空格标记`
commentstyle=\color{red!50!green!50!blue!50}, frame=lines,
%`框住代码`
rulesepcolor=\color{red!20!green!20!blue!20},
%`框线条颜色`
escapeinside=**,
%`逃逸符号自己设定符号将~*~改为自己定的符号`
xleftmargin=2em,xrightmargin=2em,
%`左右边距`
aboveskip=1em}
%`上下距离`
\end{lstlisting}

以下在正文
\verb$\begin{lstlisting}[language={[LaTeX]TeX}]$
此处粘贴代码
\verb$\end{lstlisting}$
以下为导入外部文件代码
\verb$\lstinputlisting{body/asycode/Clip.asy}$
效果如下:
\lstinputlisting{body/asycode/Clip.asy}


\subsubsection{listings 宏包设置文件注意事项}
\begin{enumerate}
  \item 外部导入文件
  \begin{latexcmd}
    \lstinputlisting{外部文本文件路径}
  \end{latexcmd}

  \item 新建 lstlisting 环境
   \begin{latexcmd}
    \lstdefinelanguage{Asymptote}{参数设置...}
    \lstalias{asy}{Asymptote}% 定义别名
    \end{latexcmd}


  \item 在 PDF 中显示的 \verb|`| 和 $'$ 复制到 WORD 中会变成 ` 和 ’ 。
  \item 中文必须用逃逸符号框起来
 % \item 代码太长可以在代码前用 \verb|\tiny \scriptsize \footnotesize| 在代码后再用\verb|\normalsize \xiaosi|等命令恢复
\end{enumerate}

\subsubsection{listings 高亮示例}

不加[],为导言区的默认语言选项设置,加[]后,语言和其他选项可重新定义。附录中有可高亮的语言列表,也可以自定语言。
如本文定义的 ASY 语言。

\begin{enumerate}

\item \LaTeX 语言
\begin{lstlisting}[language={[LaTeX]TeX}]

\usepackage{listings}
\begin{lstlisting}[language={[LaTeX]TeX}]
\begin{lstlisting}[language={[x86masm]Assembler}]
\begin{lstlisting}[language={C}]
\begin{lstlisting}[language={C++}]
\begin{lstlisting}[language={VHDL}]
\begin{lstlisting}[language={Verilog}]

\end{lstlisting}
  \item C 语言
\begin{lstlisting}[language={C++}]
for i:=maxintto 0
    do
         begin
         { do nothing }
          end;
Write('Caseinsensitive');
WritE('Pascal keywords.');
\end{lstlisting}
  \item 汇编语言
\begin{lstlisting}[language={[x86masm]Assembler}]
ORG 00H

START:MOV A,#0FFH    ;`赋初值;`
      CLR C
      MOV R2,#8

LOOP1:RRC A           ; `带进位右移;`
     MOV P1,A
     CALL DELAY
     DJNZ R2,LOOP1
     MOV R2,#7

LOOP2:RLC A          ; `带进位左移;`
      MOV P1,A
      CALL DELAY
      DJNZ R2,LOOP2
      JMP START

DELAY:MOV R3,#20     ;`延时0.2秒`
D1:   MOV R4,#20
D2:   MOV R5,#248
\end{lstlisting}
  \item ASY 语言
\begin{lstlisting}
//`背景网格`
for (int i = 0; i <= 8; ++i) {
  real x = i * cm;
  //`横线`
  draw((0,x)--(8cm,x));
  //`竖线`
  draw((x,0)--(x,8cm));
}
\end{lstlisting}

  \item Verilog 语言,注意逃逸字符不能和语言中的字符一样。
\begin{lstlisting}[language={Verilog},escapeinside=&&,basicstyle={\ttfamily}]
input [2:0] sel;
input b_in, cy;
input [7:0] acc, ram, op2, des;

output eq;
reg eq;

always @(sel or b_in or cy or acc or ram or op2 or des)
begin
  case (sel)
    `CSS_AZ : eq = (acc == 8'h00);
    `CSS_AR : eq = (acc == ram);
    `CSS_AC : eq = (acc == op2);
    `CSS_CR : eq = (op2 == ram);
    `CSS_DES : eq = (des == 8'h00);
    `CSS_CY : eq = cy;
    `CSS_BIT : eq = b_in;
    default: eq = 1'bx;
  endcase
end

endmodule
\end{lstlisting}

\end{enumerate}

也可以在环境前加上 lstset 命令
\begin{cmd}
\lstset{language={[LaTeX]TeX}}
\begin{lstlisting}
\newcommand{}
\XeLaTeX 代码
\end{lstlisting}
\end{cmd}

效果如下:

\lstset{language={[LaTeX]TeX}}
\begin{lstlisting}
\newcommand{}
\XeLaTeX 代码
\end{lstlisting}



\subsection{fancyvrb 宏包}

\index{宏包!fancyvrb}
\index{环境!Verbatim}

这个宏包功能比 listing 宏包要弱,没有高亮关键字,但在 PDF 中复制比较好,且支持中文。
这个宏包的环境,生成的代码在 PDF 中可以直接复制粘贴,不会出现多出很多空格的情况,就是参数设置每次都要写多一点。
listing 有彩色高亮等功能,但生成到 PDF 中直接复制的代码格式需调整。
所以用于讲解代码用 listing 宏包,需要粘贴复制就用 verbatim 宏包。

fancyvrb 为改进的 verbatim 宏包,可以设置抄录的各种格式,还可在脚注等环境中插入。
verb 抄录命令注:单行,不能换行,符号为不是*和空格,
且文本中没有出现的任一符号,通常用 \$ $|$ " 这些无方向字符。
抄录环境和抄录中的重音号\verb|`|和单引号\fbox{$'$}显示为英文的单引号\fbox{`}和单引号\fbox{'}
带*号的代码中空格将以$\sqcup$ 显示出来。
\begin{lstlisting}[language={[LaTeX]TeX}]

`系统抄录环境`
\begin{verbatim}
`文本`
\end{verbatim}
\begin{verbatim*}
`文本`
\end{verbatim*}

`fancyvrb抄录环境`
\begin{Verbatim}[`参数1=`,`参数2=`]
`文本`
\end{Verbatim}

`抄录命令 `
\verb`符号  文本  符号`
\verb*`符号  文本  符号`
\end{lstlisting}

\begin{verbatim}
\begin{Verbatim}[formatcom=\color{grass},frame=single,numbers=left]
字体颜色,外框,行号
\end{Verbatim}
\end{verbatim}

\subsubsection{自定义 Verbatim 环境}
参考 VIM 7.2 手册的 LATEX 代码,在文档开始前,引导区内写下如下代码
\begin{shaded}
  \begin{Verbatim}[fontsize=\tiny]
\DefineVerbatimEnvironment{VimVBshcmd}{Verbatim}
  {gobble=0,rulecolor=\color{black},formatcom=\color{blue},samepage=true,numbers=none,numbersep=0mm,
  frame=single,framerule=0.1pt,label=shell command,fontsize=\small}
\DefineVerbatimEnvironment{latexcmd}{Verbatim}
  {gobble=0,rulecolor=\color{black},formatcom=\color{blue},samepage=true,numbers=none,numbersep=0mm,
   frame=single,framerule=0.1pt,label=\LaTeX \quad 命令 ,fontsize=\small}
  \begin{VimVBshcmd}
[/home/someuser/src]$ ctags *
\end{VimVBshcmd}
\begin{latexcmd}
  \begin{Verbatim}[fontsize=\small]
\end{latexcmd}

\end{Verbatim}
\end{shaded}

\begin{VimVBshcmd}
[/home/someuser/src]$ ctags *
\end{VimVBshcmd}

\begin{latexcmd}
  \begin{Verbatim}[fontsize=\small]
\end{latexcmd}
除了在引导区,还可在文档内用\verb|\fvset| 命令来定义环境。
\begin{latexcmd}
\fvset{frame=single,numbers=left,numbersep=3pt}
\VerbatimInput{figures/frf.asy}
\end{latexcmd}
 效果如下:
\fvset{frame=single,numbers=left,numbersep=3pt}
\VerbatimInput{figures/frf.asy}
\fvset{frame=single,numbers=none,numbersep=3pt}

如果内容超出了边界,可将 xleftmargin 和 xrightmargin 设为负值来修正。


\scriptsize
\begin{shaded}
\begin{verbatim}
\begin{cmd}[label=修改MACTEX字体文件命令,xrightmargin=-0.5cm]
sudo vim /usr/local/texlive/2012/texmf-dist/tex/latex/ctex/fontset/ctex-xecjk-winfonts.def
\end{cmd}
\end{verbatim}
\end{shaded}
\normalsize
\begin{cmd}[label=修改MACTEX字体文件命令]
sudo vim /usr/local/texlive/2012/texmf-dist/tex/latex/ctex/fontset/ctex-xecjk-winfonts.def
\end{cmd}
\begin{cmd}[label=修改MACTEX字体文件命令,xrightmargin=-0.5cm]
sudo vim /usr/local/texlive/2012/texmf-dist/tex/latex/ctex/fontset/ctex-xecjk-winfonts.def
\end{cmd}

\subsection{sverb 宏包}
\index{宏包!sverb}
\index{命令!\verb$\verbinput$}

用sverb宏包可加入外部文件。\\
\fbox{\parbox{12cm}{
\verbinput{body/wangfan.txt}}}
