\subsection{图形变换}
shift,scale,rotate,fill,clip,
\subsubsection{平移,放缩,旋转}
\begin{description}
  \item[平移] shift(x,y)$*$图像
  \item[放缩] xscale(放大倍数)$*$图像: xscale(real x),yscale(real y),scale(real s),scale(real x,real y)
  \item[旋转] rotate(旋转角度, 点坐标),绕点逆时针旋转
  \item[反射] reflect(点a,点b):以 a - - b 为对称轴反射
\end{description}

\subsubsection{填充,裁剪}
\begin{itemize}
  \item add(图像,原点)
  \item add(图像,above=false)
  \item fill(封闭区域,颜色)
  \item filldraw(封闭区域,fillpen=填充颜色,drawpen=路径颜色)
  \item clip(裁剪对象(pic),裁剪路径):裁剪命令处理完后 pic 只剩下裁剪出的部分,可用 add 函数再次加入。
\end{itemize}
\textcolor[rgb]{1.00,0.00,0.50}{picture 类型是一个独立的图,用 draw,filldraw 绘制后不会显示,要显示必须用 add 命令。}


\textcolor[rgb]{0.25,0.50,0.50}{填充阴影}使用 pattern 宏包\\

\begin{minipage}{12cm}
\lstinputlisting{body/asycode/pattern.asy}
\end{minipage}


hatch(NW) 是一种西北走向的阴影斜线的图形; 用\verb$ add("name",hatch(NW))$; 命名为 name。
接下去用 \verb$pattern("name")$ 的方式把它做成一个类似与颜色的画笔。
hatch() 函数还有其他参数, 比如线的粗细, 线的间隔等.



\clearpage
