\section{索引}
\index{命令!\verb$\makeindex$}
\index{命令!\verb$\printindex$}
\index{宏包!makeidx}

\subsection{默认宏包 makeidx}
用\textcolor[rgb]{1.00,0.00,0.00}{ makeindex 文件名 }处理 idx 后缀文件,可不写后缀。
\begin{shaded}
\begin{Verbatim}
\usepackage{makeidx}
\makeindex % 激活索引命令
\index{索引分类项!索引目标内容} % 放在对应的内容后
\printindex % 输出索引章节
\end{Verbatim}
\end{shaded}
注意:\\
\begin{enumerate}
  \item !、 \verb|@| 和 \verb$|$ 在 makeindex 里是命令字符。不能放在\verb|\index{...}|内
  \item \verb|\makeindex|注释掉则不产生索引。\index{命令!\verb$\makeinedex$}
  \item !为输入分类符,在后新建一条目,最多嵌套 3 层条目。
\end{enumerate}
效果如文末。

\IndexList{命令}{\IndexList}
\subsection{xelatex 索引高级宏包 xeindex}

用法兼容 makeidx 宏包的命令。
\begin{lstlisting}[language={[LaTeX]TeX}]
\makeindex
\printindex
\index{entry}
`兼容以上命令,并增加以下命令:`
\IndexList*{name}{list of entries}
\end{lstlisting}
带 * 号为大小写敏感。