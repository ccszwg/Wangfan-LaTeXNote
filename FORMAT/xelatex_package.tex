
\usepackage{ctex}
%%\usepackage{ctexcap} % 需要将 ctexcap.sty 里与 ctex 宏包重复的部分注释,即不用加载相同包
\usepackage{relsize}                 % 调整公式字体大小:\mathsmaller, \mathlarger
%\usepackage{times}
\usepackage{fontspec,xunicode,xltxtra} % XeLaTeX相关字体字库

%%%%%%%%%%%%%%%%%%%%%%%%%%%%%%%%%%%%%%%%%%%%%%%%%%%

\usepackage{etex}  % 解决宏包 no room for 。。。的错误
%%%%%%%%%%% 版本修改记录宏包 %%%%%%%%%%%%%%%%%%%%%%
%\usepackage[nochapter]{vhistory}

\usepackage{caption2}                               % 按标准, 去掉图表号后面的:
\usepackage{lipsum}   % To generate test text 产生测试文本

%%%%%%%%%%%%%% 颜色 %%%%%%%%%%%%%%%%%%%%%%%%%%%%%%%%%%%%%%
\usepackage[table,dvipsnames,svgnames]{xcolor}
\usepackage{xxcolor}

%%%%%%%%%%%%%% PGF绘图宏包 %%%%%%%%%%%%%%%%%%%%%%%%%%%%%%%%%%%%%%

%%%%%%%%%%%%%%%%%%%%%%% pgf 绘图 %%%%%%%%%%%%%%%%%%%%%%%%
\def\pgfsysdriver{pgfsys-dvipdfmx.def} %使用dvipdfmx的引擎,原XETEX生成图形有的有错误。
\def\xcolorversion{2.00}
\usepackage[version=latest]{pgf}

\usepackage{xkeyval,calc,tikz}
%
\usetikzlibrary{
  arrows,
  calc,
  fit,
  patterns,
  plotmarks,
  shapes.geometric,
  shapes.misc,
  shapes.symbols,
  shapes.arrows,
  shapes.callouts,
  shapes.multipart,
  shapes.gates.logic.US,
  shapes.gates.logic.IEC,
  circuits.logic.US,
  circuits.logic.IEC,
  circuits.logic.CDH,
%  circuits.ee.IEC,
  datavisualization,
  datavisualization.formats.functions,
  er,
  automata,
  backgrounds,
  chains,
  topaths,
  trees,
  petri,
  mindmap,
  matrix,
  calendar,
  folding,
  fadings,
  shadings,
  spy,
  through,
  turtle,
  positioning,
  scopes,
  decorations.fractals,
  decorations.shapes,
  decorations.text,
  decorations.pathmorphing,
  decorations.pathreplacing,
  decorations.footprints,
  decorations.markings,
  shadows,
  lindenmayersystems,
  intersections,
  fixedpointarithmetic,
  fpu,
%  svg.path,
  external,
}

\tikzifexternalizing{%
}{%
\usepackage{pdfpages}
%\usepackage{vmargin}
}
%%%%%%%%%%%%%  电路宏包,更多电子元器件 %%%%%%%%%%%%%%%%%%%%%%%%%%
\usepackage[siunitx]{circuitikz}  %需加 etex package ,否则 supp-tex 有 no room for 。。。 的 error
\usepackage{tikz-timing}
\def\degr{${}^\circ$} %角度定义
%%%%%%%%%%%%%%%%%%%% 初始化
%
%\tikzset{external/system call={xelatex \tikzexternalcheckshellescape -halt-on-error-interaction=batchmode -jobname "\image" "\texsource"}}
%\tikzsetexternalprefix{figures/}% 设置图片保存路径
%\tikzexternalize %activate!


% Global styles:
\tikzset{
   every plot/.style={prefix=plots/pgf-},
   shape example/.style={
    color=black!30,
    draw=,
    fill=yellow!30,
    line width=.5cm,
    inner xsep=2.5cm,
    inner ysep=0.5cm}
}
\tikzset{
passprocess/.style={
rectangle,
draw=blue,
minimum width=50pt,
minimum height=20pt,
font=\ttfamily,
text centered
},
startstop/.style={
rectangle,%
rounded corners=5pt,%
minimum width=50pt,
minimum height=20pt,
text centered,
fill=orange,
font=\ttfamily,
draw=red
},
decision/.style={
diamond,%
shape aspect=2,%aspect value is the ratio of width and height for diamond
draw=green,%the color of line
fill=lime,%filled color
font=\ttfamily,%set font
text centered%surely you know what it means
},%here is a
line/.style = {
draw,
->,
%shorten>=2pt,
thick
}}




%\usepackage[active,tightpage]{preview}
%\setlength\PreviewBorder{5pt}%


%%%%<
%\PreviewEnvironment{tikzpicture}
%%%%>

\tikzset{
  paint/.style={draw=#1!50!black, fill=#1!50},
  information text/.style={rounded corners,fill=red!10,inner sep=0ex},
  my star/.style={decorate,decoration={shape backgrounds,shape=star},
                  star points=#1}
}


%%%%%%%%%%%%%  坐标图绘制宏包%%%%%%%%%%%%%%%%%%%%%%%%%%
\usepackage{pifont}
\usepackage{pgfplots}
\pgfplotsset{width=7cm,compat=1.4}

\usepgfplotslibrary{%
	ternary,
	smithchart,
	patchplots,
	polar,
	colormaps,
}

%%%%%%%%%%%%% 动画设置 %%%%%%%%%%%%%%%%%%%%%%%%%%

\tikzset{overlap/.style={fill=yellow!30},
    block wave/.style={thick},
    function f/.style={block wave, red!50},
    function g/.style={block wave, green!50},
    convolution/.style={block wave, blue!50},
    function g position/.style={function g, dashed, semithick},
    major tick/.style={semithick},
    axis label/.style={anchor=west},
    x tick label/.style={anchor=north, minimum width=7mm},
    y tick label/.style={anchor=east},
}
\pgfkeys{/pgf/number format/.cd,fixed,precision=1}

\pgfdeclarelayer{background}
\pgfdeclarelayer{foreground}
\pgfsetlayers{background,main,foreground}




%%%%%%%%%%%%% yellownote 边注设计 %%%%%%%%%%%%%%%%%%%%%%%%%%

\newlength{\yellownotewidth}
\setlength{\yellownotewidth}{2cm}
\newlength{\yellownoteheight}
\setlength{\yellownoteheight}{2cm}
\newcommand{\yellownote}[1]{
\marginpar{
    \vspace{-0.5\yellownoteheight}
        \begin{center}
        \begin{tikzpicture}
            \draw[white,fill=gray!25,opacity=0.75,shift={(-0.125,-0.125)}]
                (0,0) rectangle (\yellownotewidth,\yellownoteheight);
            \draw[fill=yellow!35] (0,0) rectangle (\yellownotewidth,\yellownoteheight);
            \draw[opacity=0.45,fill=gray!50] (0.7\yellownotewidth,0) --
                (0.9\yellownotewidth,0.45) -- (\yellownotewidth,0.4) -- cycle;
            \node[blue,below] at (0.5\yellownotewidth,\yellownoteheight) {
                \begin{minipage}{\yellownotewidth-1em}
                    \scriptsize\sf#1
                \end{minipage}
            };
        \end{tikzpicture}
        \end{center}
        \vspace{0.5\yellownoteheight}
    }
}

%   -   -   -   -   -   -   -   -   -   -   -   -
% Resizeable - Yellow note...
%   -   -   -   -   -   -   -   -   -   -   -   -
\newcommand{\resizeyellownote}[3]{
\setlength{\yellownotewidth}{#1cm}
\setlength{\yellownoteheight}{#2cm}
\marginpar{
    \vspace{-0.5\yellownoteheight}
        \begin{center}
        \begin{tikzpicture}
            \draw[white,fill=gray!25,opacity=0.75,shift={(-0.125,-0.125)}]
                (0,0) rectangle (\yellownotewidth,\yellownoteheight);
            \draw[fill=yellow!35] (0,0) rectangle (\yellownotewidth,\yellownoteheight);
            \draw[opacity=0.45,fill=gray!50] (0.7\yellownotewidth,0) --
                (0.9\yellownotewidth,0.45) -- (\yellownotewidth,0.4) -- cycle;
            \node[blue,below] at (0.5\yellownotewidth,\yellownoteheight) {
                \begin{minipage}{\yellownotewidth-1em}
                    \scriptsize\sf#3
                \end{minipage}
            };
        \end{tikzpicture}
        \end{center}
        \vspace{0.5\yellownoteheight}
    }
}





%%%%%%%%%%%% 合并PDF文档 与tikz extern 冲突%%%%%%%%%%%%%%%%%%%%%%
%\usepackage{pdfpages}

%%%%%%%%%%%% 图表标题格式包 %%%%%%%%%%%%%%%%%%%%%%
\usepackage[Euler]{upgreek}
\usepackage{amsmath,amsfonts,amssymb} %
\usepackage{latexsym,bm}        %公式符号
\usepackage[misc,electronic,clock]{ifsym} %电气符号
\usepackage{dingbat}
\usepackage[Omega,upmu]{gensymb}
\usepackage{wasysym}
\usepackage{marvosym}

%%%%%%%%%%%%%%%%    插图  %%%%%%%%%%%%%%%%%%%%%%%%%%%%%%%%%%%%
\usepackage{graphicx}       %插图宏包
\usepackage{wallpaper}    %绘图文绕排宏包,页面背景宏包,
\usepackage{picinpar} %

%%%%%%%%%%%%%% 彩色表格,表格线条 %%%%%%%%%%%%%%%%%%%%%%%%%%%%%%%%%%%%
\usepackage{tabu}
\usepackage{booktabs,colortbl,diagbox,longtable,multirow,tabularx,dcolumn}                         %表格粗线,斜线,彩色表格,长表格
%%%%%%%%%%%%% 页版面,边距设置 %%%%%%%%%%%%%%%%%%%%%%%%%%%%%%%%%%
\usepackage[top=2.54cm,bottom=2.54cm,left=2.15cm,right=2.5cm,includehead,includefoot]{geometry}
%上下2.54,左右2
%%%%%%%%%%% 中文书签中文复制 %%%%%%%%%%%%%%%%%%%%%%%%%%%%%%%%%%
\usepackage[colorlinks=no,
            citecolor=blue,
            linkcolor=blue,
            anchorcolor=green,
            urlcolor=blue,
%% 与attachfile2冲突           pdfauthor={wangfan},%作者
%%            pdfkeywords={latex},%关键词
%%            pdfsubject={latex},%主题
%%            pdftitle={handbook of latex},%标题
            CJKbookmarks=true,
            pdfborder={0 0 0},
            bookmarksnumbered=true,
            bookmarksopen=false,
            xetex,
            ]{hyperref}
%\usepackage{ccmap}               % 使生成的PDF文件支持复制等,对pdflatex
%
%
%%%%%%%%%%%%%%%%%%%%%%%%%%%%%%%%%%%%%%%%%%%%%%%%%%%%%%%%%%%%%%%%%%%
\usepackage{titletoc}           %目录格式包


%%%%%%%%%%%%%%%%%%%%%%%%%%%%%%%%%%%%%%%%%%%%%%%%%%%%%%%%%%%%%%%%%%%标题中文化
\usepackage[bf,small,raggedright,indentafter,pagestyles]{titlesec}
        %其中bf设置章节标题的字体为黑体,这也是默认值,可以略去。
        %此外,还可以设 为rm(罗马体), sf(无衬线体), tt(打字机体), md(中等黑度),
        %up(直立体), it(意大利斜体), sl(机械斜体), sc(小体大写字母)。
        %small设置标题字体的尺寸,还可设为big(默认), medium, tiny。
        %center使标题居中,还可以设为raggedleft(居左,默认), raggedright(居右)。
        %indentafter相当于宏包indentfirst的作用,使标题下面的第一个段落正常缩进。
        %pagestyles是申明后面要自定义页面样式。

%%%%%%%%%%%%%%%%%%%%%%%%%%%%\usepackage{tocloft}

\usepackage{fancyhdr}       %自定义页眉页脚

%%%%%%%%%%%% 抄录环境 %%%%%%%%%%%%%
\usepackage{fancyvrb,sverb}
%
% # -*- coding: utf-8 -*-
% 2010-07-15

%\usepackage[hyperref]{xcolor}


% keywords 对应 asy-keyword-name
% keywords=[2] 对应 asy-type-name
% keywords=[3] 对应 asy-function-name
% keywords=[4] 对应 asy-variable-name
\usepackage{listings}
% 语言定义
\lstdefinelanguage{Asymptote}{alsoletter={},
sensitive=true,% 大小写
keywords={and,controls,tension,atleast,curl,if,else,while,for,do,return,break,continue,struct,typedef,new,access,import,unravel,from,include,quote,static,public,private,restricted,this,explicit,true,false,null,cycle,newframe,operator},
keywords=[2]{Braid,FitResult,Label,Legend,Segment,Solution,TreeNode,abscissa,arc,arrowhead,binarytree,binarytreeNode,block,bool,bool3,bounds,bqe,circle,conic,coord,coordsys,cputime,ellipse,file,filltype,frame,grid3,guide,horner,hsv,hyperbola,indexedTransform,int,inversion,key,light,line,linefit,marginT,marker,mass,object,pair,parabola,path,path3,pen,picture,point,position,projection,real,revolution,scaleT,scientific,segment,side,slice,solution,splitface,string,surface,tensionSpecifier,ticklocate,ticksgridT,tickvalues,transform,transformation,tree,triangle,trilinear,triple,vector,vertex,void},
keywords=[3]{AND,Arc,ArcArrow,ArcArrows,Arrow,Arrows,Automatic,AvantGarde,BBox,BWRainbow,BWRainbow2,Bar,Bars,BeginArcArrow,BeginArrow,BeginBar,BeginDotMargin,BeginMargin,BeginPenMargin,Blank,Bookman,Bottom,BottomTop,Bounds,Break,Broken,BrokenLog,CLZ,CTZ,Ceil,Circle,CircleBarIntervalMarker,Cos,Courier,CrossIntervalMarker,DefaultFormat,DefaultLogFormat,Degrees,Dir,DotMargin,DotMargins,Dotted,Draw,Drawline,Embed,EndArcArrow,EndArrow,EndBar,EndDotMargin,EndMargin,EndPenMargin,Fill,FillDraw,Floor,Format,Full,Gaussian,Gaussrand,Gaussrandpair,Gradient,Grayscale,Helvetica,Hermite,HookHead,InOutTicks,InTicks,Jn,Label,Landscape,Left,LeftRight,LeftTicks,Legend,Linear,Link,Log,LogFormat,Margin,Margins,Mark,MidArcArrow,MidArrow,NOT,NewCenturySchoolBook,NoBox,NoMargin,NoModifier,NoTicks,NoTicks3,NoZero,NoZeroFormat,None,OR,OmitFormat,OmitTick,OmitTickInterval,OmitTickIntervals,OutTicks,Ox,Oy,Palatino,PaletteTicks,Pen,PenMargin,PenMargins,Pentype,Portrait,RadialShade,RadialShadeDraw,Rainbow,Range,Relative,Right,RightTicks,Rotate,Round,SQR,Scale,ScaleX,ScaleY,ScaleZ,Seascape,Segment,Shift,Sin,Slant,Spline,StickIntervalMarker,Straight,Symbol,Tan,TeXify,Ticks,Ticks3,TildeIntervalMarker,TimesRoman,Top,TrueMargin,UnFill,UpsideDown,Wheel,X,XEquals,XOR,XY,XYEquals,XYZero,XYgrid,XZEquals,XZZero,XZero,XZgrid,Y,YEquals,YXgrid,YZ,YZEquals,YZZero,YZero,YZgrid,Yn,Z,ZX,ZXgrid,ZYgrid,ZapfChancery,ZapfDingbats,_begingroup3,_cputime,_draw,_eval,_image,_labelpath,_projection,_strokepath,_texpath,aCos,aSin,aTan,abort,abs,accel,acos,acosh,acot,acsc,activatequote,add,addArrow,addMargins,addSaveFunction,addnode,addnodes,addpenarc,addpenline,adjust,alias,align,all,altitude,angabscissa,angle,angpoint,animate,annotate,anticomplementary,antipedal,apply,approximate,arc,arcarrowsize,arccircle,arcdir,arcfromcenter,arcfromfocus,arclength,arcnodesnumber,arcpoint,arcsubtended,arcsubtendedcenter,arctime,arctopath,array,arrow,arrow2,arrowbase,arrowbasepoints,arrowsize,asec,asin,asinh,ask,assert,asy,asycode,asydir,asyfigure,asyfilecode,asyinclude,asywrite,atan,atan2,atanh,atbreakpoint,atexit,atime,attach,attract,atupdate,autoformat,autoscale,autoscale3,axes,axes3,axialshade,axis,axiscoverage,azimuth,babel,background,bangles,bar,barmarksize,barsize,basealign,baseline,bbox,beep,begin,beginclip,begingroup,beginpoint,between,bevel,bezier,bezierP,bezierPP,bezierPPP,bezulate,bibliography,bibliographystyle,binarytree,binarytreeNode,binomial,binput,bins,bisector,bisectorpoint,bispline,blend,blockconnector,boutput,box,bqe,breakpoint,breakpoints,brick,buildRestoreDefaults,buildRestoreThunk,buildcycle,bulletcolor,byte,canonical,canonicalcartesiansystem,cartesiansystem,case1,case2,case3,case4,cbrt,cd,ceil,center,centerToFocus,centroid,cevian,change2,changecoordsys,checkSegment,checkconditionlength,checker,checkincreasing,checklengths,checkposition,checktriangle,choose,circle,circlebarframe,circlemarkradius,circlenodesnumber,circumcenter,circumcircle,clamped,clear,clip,clipdraw,close,cmyk,code,colatitude,collect,collinear,color,colorless,colors,colorspace,comma,compassmark,complement,complementary,concat,concurrent,cone,conic,conicnodesnumber,conictype,conj,connect,connected,connectedindex,containmentTree,contains,contour,contour3,contouredges,controlSpecifier,convert,coordinates,coordsys,copy,cos,cosh,cot,countIntersections,cputime,crop,cropcode,cross,crossframe,crosshatch,crossmarksize,csc,cubicroots,curabscissa,curlSpecifier,curpoint,currentarrow,currentexitfunction,currentmomarrow,currentpolarconicroutine,curve,cut,cutafter,cutbefore,cyclic,cylinder,deactivatequote,debugger,deconstruct,defaultdir,defaultformat,defaultpen,defined,degenerate,degrees,delete,deletepreamble,determinant,diagonal,diamond,diffdiv,dir,dirSpecifier,dirtime,display,distance,divisors,do_overpaint,dot,dotframe,dotsize,downcase,draw,drawAll,drawDoubleLine,drawFermion,drawGhost,drawGluon,drawMomArrow,drawPRCcylinder,drawPRCdisk,drawPRCsphere,drawPRCtube,drawPhoton,drawScalar,drawVertex,drawVertexBox,drawVertexBoxO,drawVertexBoxX,drawVertexO,drawVertexOX,drawVertexTriangle,drawVertexTriangleO,drawVertexX,drawarrow,drawarrow2,drawline,drawtick,duplicate,elle,ellipse,ellipsenodesnumber,embed,embed3,empty,enclose,end,endScript,endclip,endgroup,endgroup3,endl,endpoint,endpoints,eof,eol,equation,equations,erase,erasestep,erf,erfc,error,errorbar,errorbars,eval,excenter,excircle,exit,exitXasyMode,exitfunction,exp,expfactors,expi,expm1,exradius,extend,extension,extouch,fabs,factorial,fermat,fft,fhorner,figure,file,filecode,fill,filldraw,filloutside,fillrule,filltype,find,finite,finiteDifferenceJacobian,firstcut,firstframe,fit,fit2,fixedscaling,floor,flush,fmdefaults,fmod,focusToCenter,font,fontcommand,fontsize,foot,format,frac,frequency,fromCenter,fromFocus,fspline,functionshade,gamma,generate_random_backtrace,generateticks,gergonne,getc,getint,getpair,getreal,getstring,gettriple,gluon,gouraudshade,graph,graphic,gray,grestore,grid,grid3,gsave,halfbox,hatch,hdiffdiv,hermite,hex,histogram,history,hline,hprojection,hsv,hyperbola,hyperbolanodesnumber,hyperlink,hypot,identity,image,incenter,incentral,incircle,increasing,incrementposition,indexedTransform,indexedfigure,initXasyMode,initdefaults,input,inradius,insert,inside,integrate,interactive,interior,interp,interpolate,intersect,intersection,intersectionpoint,intersectionpoints,intersections,intouch,inverse,inversion,invisible,is3D,isCCW,isDuplicate,isogonal,isogonalconjugate,isotomic,isotomicconjugate,isparabola,italic,item,key,kurtosis,kurtosisexcess,label,labelaxis,labelmargin,labelpath,labels,labeltick,labelx,labelx3,labely,labely3,labelz,labelz3,lastcut,latex,latitude,latticeshade,layer,layout,ldexp,leastsquares,legend,legenditem,length,lexorder,lift,light,limits,line,linear,linecap,lineinversion,linejoin,linemargin,lineskip,linetype,linewidth,link,list,lm_enorm,lm_evaluate_default,lm_lmdif,lm_lmpar,lm_minimize,lm_print_default,lm_print_quiet,lm_qrfac,lm_qrsolv,locale,locate,locatefile,location,log,log10,log1p,logaxiscoverage,longitude,lookup,magnetize,makeNode,makedraw,makepen,map,margin,markangle,markangleradius,markanglespace,markarc,marker,markinterval,marknodes,markrightangle,markuniform,mass,masscenter,massformat,math,max,max3,maxbezier,maxbound,maxcoords,maxlength,maxratio,maxtimes,mean,medial,median,midpoint,min,min3,minbezier,minbound,minipage,minratio,mintimes,miterlimit,momArrowPath,momarrowsize,monotonic,multifigure,nativeformat,natural,needshipout,newl,newpage,newslide,newton,newtree,nextframe,nextnormal,nextpage,nib,nodabscissa,none,norm,normalvideo,notaknot,nowarn,numberpage,nurb,object,offset,onpath,opacity,opposite,orientation,orig_circlenodesnumber,orig_circlenodesnumber1,orig_draw,orig_ellipsenodesnumber,orig_ellipsenodesnumber1,orig_hyperbolanodesnumber,orig_parabolanodesnumber,origin,orthic,orthocentercenter,outformat,outline,outname,outprefix,output,overloadedMessage,overwrite,pack,pad,pairs,palette,parabola,parabolanodesnumber,parallel,parallelogram,partialsum,path,path3,pattern,pause,pdf,pedal,periodic,perp,perpendicular,perpendicularmark,phantom,phi1,phi2,phi3,photon,piecewisestraight,point,polar,polarconicroutine,polargraph,polygon,postcontrol,postscript,pow10,ppoint,prc,prc0,precision,precontrol,prepend,print_random_addresses,project,projection,purge,pwhermite,quadrant,quadraticroots,quantize,quarticroots,quotient,radialshade,radians,radicalcenter,radicalline,radius,rand,randompath,rd,readline,realmult,realquarticroots,rectangle,rectangular,rectify,reflect,relabscissa,relative,relativedistance,reldir,relpoint,reltime,remainder,remark,removeDuplicates,rename,replace,report,resetdefaultpen,restore,restoredefaults,reverse,reversevideo,rf,rfind,rgb,rgba,rgbint,rms,rotate,rotateO,rotation,round,roundbox,roundedpath,roundrectangle,same,samecoordsys,sameside,sample,save,savedefaults,saveline,scale,scale3,scaleO,scaleT,scaleless,scientific,search,searchindex,searchtree,sec,secondaryX,secondaryY,seconds,section,sector,seek,seekeof,segment,sequence,setcontour,setpens,sgn,sgnd,sharpangle,sharpdegrees,shift,shiftless,shipout,shipout3,show,side,simeq,simpson,sin,single,sinh,size,size3,skewness,skip,slant,sleep,slope,slopefield,solve,solveBVP,sort,sourceline,sphere,split,sqrt,square,srand,standardizecoordsys,startScript,startTrembling,stdev,step,stickframe,stickmarksize,stickmarkspace,stop,straight,straightness,string,stripdirectory,stripextension,stripfile,stripsuffix,strokepath,subdivide,subitem,subpath,substr,sum,surface,symmedial,symmedian,system,tab,tableau,tan,tangent,tangential,tangents,tanh,tell,tensionSpecifier,tensorshade,tex,texcolor,texify,texpath,texpreamble,texreset,texshipout,texsize,textpath,thick,thin,tick,tickMax,tickMax3,tickMin,tickMin3,ticklabelshift,ticklocate,tildeframe,tildemarksize,tile,tiling,time,times,title,titlepage,topbox,transform,transformation,transpose,tremble,trembleFuzz,tremble_circlenodesnumber,tremble_circlenodesnumber1,tremble_draw,tremble_ellipsenodesnumber,tremble_ellipsenodesnumber1,tremble_hyperbolanodesnumber,tremble_marknodes,tremble_markuniform,tremble_parabolanodesnumber,triangle,triangleAbc,triangleabc,triangulate,tricoef,tridiagonal,trilinear,trim,trueMagnetize,truepoint,tube,uncycle,unfill,uniform,unique,unit,unitrand,unitsize,unityroot,unstraighten,upcase,updatefunction,uperiodic,upscale,uptodate,usepackage,usersetting,usetypescript,usleep,value,variance,variancebiased,vbox,vector,vectorfield,verbatim,view,vline,vperiodic,vprojection,warn,warning,windingnumber,write,xaxis,xaxis3,xaxis3At,xaxisAt,xequals,xinput,xlimits,xoutput,xpart,xscale,xscaleO,xtick,xtick3,xtrans,yaxis,yaxis3,yaxis3At,yaxisAt,yequals,ylimits,ypart,yscale,yscaleO,ytick,ytick3,ytrans,zaxis3,zaxis3At,zero,zero3,zlimits,zpart,ztick,ztick3,ztrans},
keywords=[4]{AliceBlue,Align,Allow,AntiqueWhite,Apricot,Aqua,Aquamarine,Aspect,Azure,BeginPoint,Beige,Bisque,Bittersweet,Black,BlanchedAlmond,Blue,BlueGreen,BlueViolet,Both,Break,BrickRed,Brown,BurlyWood,BurntOrange,CCW,CW,CadetBlue,CarnationPink,Center,Centered,Cerulean,Chartreuse,Chocolate,Coeff,Coral,CornflowerBlue,Cornsilk,Crimson,Crop,Cyan,Dandelion,DarkBlue,DarkCyan,DarkGoldenrod,DarkGray,DarkGreen,DarkKhaki,DarkMagenta,DarkOliveGreen,DarkOrange,DarkOrchid,DarkRed,DarkSalmon,DarkSeaGreen,DarkSlateBlue,DarkSlateGray,DarkTurquoise,DarkViolet,DeepPink,DeepSkyBlue,DefaultHead,DimGray,DodgerBlue,Dotted,Down,Draw,E,ENE,EPS,ESE,E_Euler,E_PC,E_RK2,E_RK3BS,Emerald,EndPoint,Euler,Fill,FillDraw,FireBrick,FloralWhite,ForestGreen,Fuchsia,Gainsboro,GhostWhite,Gold,Goldenrod,Gray,Green,GreenYellow,Honeydew,HookHead,Horizontal,HotPink,I,IgnoreAspect,IndianRed,Indigo,Ivory,JOIN_IN,JOIN_OUT,JungleGreen,Khaki,LM_DWARF,LM_MACHEP,LM_SQRT_DWARF,LM_SQRT_GIANT,LM_USERTOL,Label,Lavender,LavenderBlush,LawnGreen,Left,LeftJustified,LeftSide,LemonChiffon,LightBlue,LightCoral,LightCyan,LightGoldenrodYellow,LightGreen,LightGrey,LightPink,LightSalmon,LightSeaGreen,LightSkyBlue,LightSlateGray,LightSteelBlue,LightYellow,Lime,LimeGreen,Linear,Linen,Log,Logarithmic,Magenta,Mahogany,Mark,MarkFill,Maroon,Max,MediumAquamarine,MediumBlue,MediumOrchid,MediumPurple,MediumSeaGreen,MediumSlateBlue,MediumSpringGreen,MediumTurquoise,MediumVioletRed,Melon,MidPoint,MidnightBlue,Min,MintCream,MistyRose,Moccasin,Move,MoveQuiet,Mulberry,N,NE,NNE,NNW,NW,NavajoWhite,Navy,NavyBlue,NoAlign,NoCrop,NoFill,NoSide,OldLace,Olive,OliveDrab,OliveGreen,Orange,OrangeRed,Orchid,Ox,Oy,PC,PaleGoldenrod,PaleGreen,PaleTurquoise,PaleVioletRed,PapayaWhip,Peach,PeachPuff,Periwinkle,Peru,PineGreen,Pink,Plum,PowderBlue,ProcessBlue,Purple,RK2,RK3,RK3BS,RK4,RK5,RK5DP,RK5F,RawSienna,Red,RedOrange,RedViolet,Rhodamine,Right,RightJustified,RightSide,RosyBrown,RoyalBlue,RoyalPurple,RubineRed,S,SE,SSE,SSW,SW,SaddleBrown,Salmon,SandyBrown,SeaGreen,Seashell,Sepia,Sienna,Silver,SimpleHead,SkyBlue,SlateBlue,SlateGray,Snow,SpringGreen,SteelBlue,Suppress,SuppressQuiet,Tan,TeXHead,Teal,TealBlue,Thistle,Ticksize,Tomato,Turquoise,UnFill,Up,VERSION,Value,Vertical,Violet,VioletRed,W,WNW,WSW,Wheat,White,WhiteSmoke,WildStrawberry,XYAlign,YAlign,Yellow,YellowGreen,YellowOrange,addpenarc,addpenline,align,allowstepping,angularsystem,animationdelay,appendsuffix,arcarrowangle,arcarrowfactor,arrow2sizelimit,arrowangle,arrowbarb,arrowdir,arrowfactor,arrowhookfactor,arrowlength,arrowsizelimit,arrowtexfactor,authorpen,axis,axiscoverage,axislabelfactor,background,backgroundcolor,backgroundpen,barfactor,barmarksizefactor,basealign,baselinetemplate,beveljoin,bigvertexpen,bigvertexsize,black,blue,bm,bottom,bp,brown,bullet,byfoci,byvertices,camerafactor,chartreuse,circlemarkradiusfactor,circlenodesnumberfactor,circleprecision,circlescale,cm,codefile,codepen,codeskip,colorPen,coloredNodes,coloredSegments,conditionlength,conicnodesfactor,count,cputimeformat,crossmarksizefactor,currentcoordsys,currentlight,currentpatterns,currentpen,currentpicture,currentposition,currentprojection,curvilinearsystem,cuttings,cyan,darkblue,darkbrown,darkcyan,darkgray,darkgreen,darkgrey,darkmagenta,darkolive,darkred,dashdotted,dashed,datepen,dateskip,debuggerlines,debugging,deepblue,deepcyan,deepgray,deepgreen,deepgrey,deepmagenta,deepred,default,defaultControl,defaultS,defaultbackpen,defaultcoordsys,defaultexcursion,defaultfilename,defaultformat,defaultmassformat,defaultpen,diagnostics,differentlengths,dot,dotfactor,dotframe,dotted,doublelinepen,doublelinespacing,down,duplicateFuzz,edge,ellipsenodesnumberfactor,eps,epsgeo,epsilon,evenodd,extendcap,exterior,fermionpen,figureborder,figuremattpen,firstnode,firststep,foregroundcolor,fuchsia,fuzz,gapfactor,ghostpen,gluonamplitude,gluonpen,gluonratio,gray,green,grey,hatchepsilon,havepagenumber,heavyblue,heavycyan,heavygray,heavygreen,heavygrey,heavymagenta,heavyred,hline,hwratio,hyperbola,hyperbolanodesnumberfactor,identity4,ignore,inXasyMode,inch,inches,includegraphicscommand,inf,infinity,institutionpen,intMax,intMin,interior,invert,invisible,itempen,itemskip,itemstep,labelmargin,landscape,lastnode,left,legendhskip,legendlinelength,legendmargin,legendmarkersize,legendmaxrelativewidth,legendvskip,lightblue,lightcyan,lightgray,lightgreen,lightgrey,lightmagenta,lightolive,lightred,lightyellow,line,linemargin,lm_infmsg,lm_shortmsg,longdashdotted,longdashed,magenta,magneticPoints,magneticRadius,mantissaBits,markangleradius,markangleradiusfactor,markanglespace,markanglespacefactor,mediumblue,mediumcyan,mediumgray,mediumgreen,mediumgrey,mediummagenta,mediumred,mediumyellow,middle,minDistDefault,minblockheight,minblockwidth,mincirclediameter,minipagemargin,minipagewidth,minvertexangle,miterjoin,mm,momarrowfactor,momarrowlength,momarrowmargin,momarrowoffset,momarrowpen,monoPen,morepoints,nCircle,newbulletcolor,ngraph,nil,nmesh,nobasealign,nodeMarginDefault,nodesystem,nomarker,nopoint,noprimary,nullpath,nullpen,numarray,ocgindex,oldbulletcolor,olive,orange,origin,overpaint,page,pageheight,pagemargin,pagenumberalign,pagenumberpen,pagenumberposition,pagewidth,paleblue,palecyan,palegray,palegreen,palegrey,palemagenta,palered,paleyellow,parabolanodesnumberfactor,perpfactor,phi,photonamplitude,photonpen,photonratio,pi,pink,plain,plus,preamblenodes,pt,purple,r3,r4a,r4b,randMax,realDigits,realEpsilon,realMax,realMin,red,relativesystem,reverse,right,roundcap,roundjoin,royalblue,salmon,saveFunctions,scalarpen,sequencereal,settings,shipped,signedtrailingzero,solid,springgreen,sqrtEpsilon,squarecap,squarepen,startposition,stdin,stdout,stepfactor,stepfraction,steppagenumberpen,stepping,stickframe,stickmarksizefactor,stickmarkspacefactor,textpen,ticksize,tildeframe,tildemarksizefactor,tinv,titlealign,titlepagepen,titlepageposition,titlepen,titleskip,top,trailingzero,treeLevelStep,treeMinNodeWidth,treeNodeStep,trembleAngle,trembleFrequency,trembleRandom,tremblingMode,undefined,unitcircle,unitsquare,up,urlpen,urlskip,version,vertexpen,vertexsize,viewportmargin,viewportsize,vline,white,wye,xformStack,yellow,ylabelwidth,zerotickfuzz,zerowinding},
% otherkeywords={!,@,\$,\%,+,-,^,=,>,<,->,
% --,..,**,::,\@\@,\$\$,---,...},% 运算符等,但小心会与注释冲突
morecomment=[l]{//},% 注释
morecomment=[s]{/*}{*/},% 注释
morestring=[b]",% 字符串
morestring=[b]',% 字符串
}
% 定义别名
\lstalias{asy}{Asymptote}
\lstset{%extendedchars=false,% 解决中文跨页出错的问题;对 xetex 无用
language={Asymptote},
basewidth={.5em},
basicstyle={\ttfamily},%如果用 rmfamily 和 sffamily 在PDF中复制格式会多出很多空格
keywordstyle={\color{keyword}},
keywordstyle=[2]{\color{type}},
keywordstyle=[3]{\color{function-name}},
keywordstyle=[4]{\color{variable-name}},
commentstyle={\color{comment}},
stringstyle={\color{string}},
xleftmargin={2em},
xrightmargin={2em},
tabsize=8,
backgroundcolor={\color{shadecolor}},
% numbers=left,
numberstyle=\tiny,
showstringspaces=false, %不显示空格标记
stepnumber=1,
escapeinside=``,
numbersep=5pt}
%
\lstdefinestyle{lesscolor}{keywordstyle={\color{keyword!50!black}},
keywordstyle=[2]{\color{type!50!black}},
keywordstyle=[3]{\color{function-name!50!black}},
keywordstyle=[4]{\color{variable-name!50!black}},
commentstyle={\color{comment!50!black}},
stringstyle={\color{string!50!black}}}
%
%\def\oldvert{|} % 保存字符 | 的旧定义(其 catcode 在此定义读入时已确定)
%\lstMakeShortInline[style=lesscolor]\|
%
%
\def\inlinecode{\expandafter\lstinline[style=lesscolor]}

\endinput


 %listings语法高亮设置

%\usepackage{fancybox} %与framed宏包冲突
%%%%%%%%%%%% 盒子环境 %%%%%%%%%%%%%
\usepackage{framed}


%%%%%%%%%%%%% ASY绘图宏包 %%%%%%%%%%%%%%%%%%%%%%
\usepackage{asymptote}

%%%%%%%%%%%%% SHAPE宏包 %%%%%%%%%%%%%%%%%%%%%%
\usepackage{shapepar}

%%%%%%%%%%%%% 图片放置宏包 不放在文字前面 %%%%%%%%%%%%%%%%%%%%%%
\usepackage{flafter,float}

%%%%%%%%%%%%% 下划线宏包 %%%%%%%%%%%%%%%%%%%%%%
 \usepackage[normalem]{ulem}%`加入宏包`
 \usepackage{CJKfntef} %汉字下划线宏包

%%%%%%%%%%%%%% 页码宏包 (与动画宏包冲突)%%%%%%%%%%%%%%%%%%%%%%
\usepackage{lastpage}

%%%%%%%%%%%%% 动画宏包 %%%%%%%%%%%%%%%%%%%%%%
\usepackage{animate} % 与 tikz 部分宏包冲突

%%%%%%%%%%%% 行号宏包 %%%%%%%%%%%%%%%%%%%%%%
\usepackage[left]{lineno} %与 tikz 宏包冲突

%%%%%%%%%%%%% 视频宏包 %%%%%%%%%%%%%%%%%%%%%%
\usepackage{movie15} % 与 tikz 部分宏包冲突

%
%%%%%%%%%%%%% 时间宏包 %%%%%%%%%%%%%%%%%%%%%%
%%%\usepackage{tdclock}  与 pdfcomment 冲突


%%%%%%%%%%%%% 短列表宏包 %%%%%%%%%%%%%%%%%%%%%%
%\usepackage{shortlst}
%
%%%%%%%%%%%%% 列表编号宏包 %%%%%%%%%%%%%%%%%%%%%%
\usepackage{enumerate}



%%%%%%%%%%%% 脚注尾注宏包 %%%%%%%%%%%%%%%%%%%%%%
\usepackage{threeparttable,endnotes}

%%%%%%%%%%%% 索引表 %%%%%%%%%%%%%%%
\usepackage{makeidx}\makeindex

%%%%%%%%%%% 索引宏包 %%%%%%%%%%%%%%%%%%%%%%
%\usepackage{xesearch}
%\usepackage{xeindex}\makeindex


%%%%%%%%%%%% 引用包 %%%%%%%%%%%%%%%
\usepackage{cite}  %实现[1-4]方式引用多个参考文献

%%%%%%%%%%%% 双栏排版宏包 %%%%%%%%%%%%%%%
\usepackage{flushend,cuted}
\usepackage{multicol} %多栏排版
%
%%%%%%%%%%%%%% 生成HTML宏包 %%%%%%%%%%%%%%%%%%%%%%
%%\usepackage{html,epsf}


%%%%%%%%%%% 附件宏包 %%%%%%%%%%%%%%%%%%%%%%
\usepackage{attachfile2}

%%%%%%%%%%% 目录结构图宏包 %%%%%%%%%%%%%%%%%%%%%%
\usepackage{dirtree}

%%%%%%%%%%% 柱状图宏包 %%%%%%%%%%%%%%%%%%%%%%
\usepackage{bardiag}

%%%%%%%%%%% 书签宏包 %%%%%%%%%%%%%%%%%%%%%%
\usepackage[open,openlevel=0,atend]{bookmark}



%%%%%%%%%%% pdf 注释宏包 %%%%%%%%%%%%%%%%%%%%%%
%\usepackage[subject={tex},author={wangfan},dvipdfmx,version=1]{pdfcomment}
