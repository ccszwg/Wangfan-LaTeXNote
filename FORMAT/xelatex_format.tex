
%%%%%%%%%%%%%%%% 重定义字号命令 %%%%%%%%%%%%%%%%%%%%%%%%%%%%%%%%%%%%%
\XeTeXlinebreaklocale "zh"
\XeTeXlinebreakskip = 0pt plus 1pt minus 0.1pt  %%這兩行一行使中文能自動換行

%%-----------------------------------------------------------------------------------------



\newcommand{\song}{\songti}
\newcommand{\fs}{\fangsong}
\newcommand{\hei}{\heiti}
\newcommand{\kai}{\kaishu}
\newcommand{\li}{\lishu}
\newcommand{\you}{\youyuan}



\newcommand{\yihao}{\fontsize{26pt}{36pt}\selectfont}    % 一号, 1.4倍行距
\newcommand{\erhao}{\fontsize{22pt}{28pt}\selectfont}    % 二号, 1.25倍行距
\newcommand{\xiaoer}{\fontsize{18pt}{18pt}\selectfont}    % 小二, 单倍行距
\newcommand{\sanhao}{\fontsize{16pt}{24pt}\selectfont}    % 三号, 1.5倍行距
%\newcommand{\xiaosan}{\fontsize{15pt}{22pt}\selectfont}    % 小三, 1.5倍行距
\newcommand{\xiaosan}{\fontsize{15pt}{36pt}\selectfont}    % 小三, 论文行距
\newcommand{\sihao}{\fontsize{14pt}{21pt}\selectfont}    % 四号, 1.5倍行距
\newcommand{\banxiaosi}{\fontsize{13pt}{19.5pt}\selectfont}    % 半小四, 1.5倍行距
\newcommand{\xiaosi}{\fontsize{12pt}{18pt}\selectfont}    % 小四, 1.5倍行距
\newcommand{\dawuhao}{\fontsize{11pt}{11pt}\selectfont}    % 大五号, 单倍行距
\newcommand{\wuhao}{\fontsize{10.5pt}{10.5pt}\selectfont}    % 五号, 单倍行距

%
%\setmainfont{Times New Roman} %设置默认英文字体。
%%\setmainfont{TeX Gyre Pagella}
%\setsansfont{TeX Gyre Adventor}
%\setmonofont{TeX Gyre Cursor}
%%%%%%%%%%%%%%%%%%%%%%%%%%%%%%%%%%%%%%%%%%%%%%%%%%%%%%%%%%%


%%%%%%%%%%%%%% 颜色 %%%%%%%%%%%%%%%%%%%%%%%%%%%%%%%%%%%%%%

\definecolor{lightgray}{rgb}{0.75,0.75, 0.75}
\definecolor{grass}{rgb}{0.00,0.50,0.25}   %用于程序语言
\definecolor{lightgreen}{rgb}{0.93,1.00,0.93}

\definecolor{lightred}{rgb}{1.00,0.50,0.50}
\definecolor{deepyellow}{rgb}{0.91,0.91,0.00}
\definecolor{lightyellow}{rgb}{1,1,0.9}%浅黄,适合作代码框背景
\definecolor{deepgreen}{rgb}{0.00,0.50,0.00} %深绿,适合作代码注释
\definecolor{deepblue}{rgb}{0.00,0.40,0.40}
\definecolor{lightblue}{rgb}{0.50,0.50,1.00} %浅蓝
\definecolor{whiteblue}{rgb}{0.82,0.82,1.00} %蓝白


\definecolor{shadecolor}{rgb}{0.92,0.92,0.92} %文本背景色
\definecolor{backcolor}{rgb}{1,1,0.95} %米黄色背景


\definecolor{bg-color}{rgb}{0.96,1,0.95}
\definecolor{shadecolor}{rgb}{0.96,1,0.95} %文本背景色
\definecolor{info}{rgb}{1.00,0.50,0.50} %粉色
\definecolor{txt-color}{HTML}{000000}
\definecolor{builtin}{HTML}{DA70D6}
\definecolor{comment}{HTML}{B22222}
\definecolor{comment-delimiter}{HTML}{B22222}
\definecolor{constant}{HTML}{5F9EA0}
\definecolor{function-name}{HTML}{0000FF}
\definecolor{keyword}{HTML}{a020F0}
\definecolor{string}{HTML}{BC8F8F}
\definecolor{type}{HTML}{228B22}
\definecolor{variable-name}{HTML}{B8860B}
\definecolor{brick}{HTML}{7B0C00}

%%%%%%%%%%%%%%% 背景色 %%%%%%%%%%%%%%%%%%%%%%%%%%%%%%%%%%%%%%%
\pagecolor{backcolor}

%%%%%%%%%%%%% 超链接 %%%%%%%%%%%%%%%%%%%%%%%%%%%%%

\newcommand{\upcite}[1]{\textsuperscript{\textsuperscript{\cite{#1}}}} %参考文献上标\upcite



%%%%%%%%%%%%%%%%浮动图形和表格标题样式%%%%%%%%%%%%

\renewcommand{\captionlabeldelim}{\hspace{0em}}     % 按清华标准, 图表标题字体为11pt, 这里写作大五号
\renewcommand{\captionfont}{\wuhao\kai}



%%%%%%%%%%%%%%%%%%%%% 图路径设置 %%%%%%%%%%%%%%%%%%%
\graphicspath{{figures/}}


%%%%%%%%%%%%%%% 页眉页脚 %%%%%%%%%%%%%%%%%%%%%%%%%


\pagestyle{fancy}           %\fancyhf{}为清空页眉页脚,\rhead{} \lhead{} \lfoot{} \cfoot{} \rfoot{\thepage} 命令设置
\addtolength{\headsep}{-0.1cm}          %页眉位置
\addtolength{\footskip}{0.4cm}         %页脚位置



%%%%%%%%%%%%%%%%%% 定义段落章节的标题和目录项的格式 %%%%%%%%%%%%%%%%%%%%%%%%%%%%%%%%


\titleformat{\chapter}[hang]{\centering\LARGE\bfseries}{\chaptername}{1em}{}
   %其中\chapter可以换为\section, \subsection等,设置节、小节等标题的格式。
    %hang 表示标题头与标题内容在同一行,是默认值。而book类默认的章标题是标题头与标题内容放在两个段落,
    %对应于display选项。此外还有block,runin, leftmargin, rightmargin, frame, wrap等选项,一般不大用到。
    % \centering\LARGE\bfseries这一块是设置标题的排版格式,这里设置为居中、 \LARGE尺寸和黑体。
    %后面紧跟的是标题头的定义。book类里的标题头是英文,需要改成中文。如果希望改成“第一章”这样的格式,    %则应先引用CJKnumb宏包,它提供了把阿拉伯数字转换成中文数字的命令。然后定义

\setcounter{secnumdepth}{4} \setcounter{tocdepth}{4}
\titlespacing{\chapter}{0pt}{2.4ex}{2.4em}          % ex相当于当前字母尺寸小写字母x的高度,em相当于大写M的宽度

\titleformat{\section}[hang]{\color{blue}\large\bfseries}{\thesection}{1em}{}
\titleformat{\subsection}[hang]{\color{grass}\large\bfseries}{\thesubsection}{1em}{}
\titleformat{\subsubsection}[hang]{\color{lightblue}\large\bfseries}{\thesubsubsection}{1em}{}
%%%%%%%%%%%%%%%%%%%%%%%%%%% 定义目录格式 %%%%%%%%%%%%%%%%%%%%%%%%%%%%%%%%%%

\titlecontents{chapter}
[0em]
{\color{blue}\heiti\large\heiti\addvspace{1.5ex}}        %[0em]为目录距左边的距离\addvspace为章与章之间的行距
{\xiaosi\thecontentslabel{~~~}}
{}          %{2em}为章号距章标题的距离{}为章标题前的内容
{\titlerule*[0.5pc]{.}\contentspage}[\addvspace{0.5ex}] %0.5ex为章到下节的距离



\titlecontents{section}
[2em] {\color{blue}\normalsize\addvspace{0.5ex}}
{\thecontentslabel\hspace*{1em}} {\hspace*{-2.3em}}
{\titlerule*[0.4pc]{.}\contentspage}


\titlecontents{subsection}
[4em] {\color{grass}\normalsize\addvspace{0.5ex}}
{\thecontentslabel\hspace*{1em}} {\hspace*{-2.3em}}
{\titlerule*[0.4pc]{.}\contentspage}

\titlecontents{subsubsection}
[6em] {\color{grass}\normalsize\addvspace{0.5ex}}
{\thecontentslabel\hspace*{1em}} {\hspace*{-2.3em}}
{\titlerule*[0.4pc]{.}\contentspage}


%%%%%%%%%%%%%%%%%% 图表目录合并 %%%%%%%%%%%%%%%%%%%%%%%%%%%%%%%



\titlecontents{figure} [0em]
{\xiaosi\song\addvspace{-0.5ex}}      %行距
{\color{grass}\thecontentslabel{~~~}}{}         %图号与标题间距
{\titlerule*[0.5pc]{.}\hspace*{.1em}\contentspage}[\large\addvspace{1.8ex}]

\titlecontents{table}[0em]
{\xiaosi\song\addvspace{-0.5ex}}
{\color{grass}\thecontentslabel{~~~}}{}
{\titlerule*[0.5pc]{.}\hspace*{.1em}\contentspage}[\large\addvspace{1.8ex}]



%%%%%%%%%%%%%%%%% 粘贴代码宏包设置 %%%%%%%%%%%%%%%%%%%%%%%%%%%%%%%

%
%
%\lstset{
%numbers=left, numberstyle=\footnotesize\itshape\mdseries,%行号显示在代码左边,行号字体为small
%basicstyle=\small\rmfamily\upshape\mdseries,                    %代码的字号
%backgroundcolor=\color{lightgreen},            %背景为灰色
%keywordstyle=\color{blue!70}\mdseries\rmfamily,               %关键字高亮加粗
%stringstyle=\rmfamily\upshape,
%showstringspaces=false,         %不显示空格标记
%commentstyle=\color{red!50!green!50!blue!50},
%frame=single, framerule=0pt,                                 %框住代码
%rulesepcolor=\color{red!20!green!20!blue!20},   %框线条颜色
%escapeinside=``,                                %框线条颜色 逃逸符号
%xleftmargin=2em,xrightmargin=2em,               %左右边距与上下边跑
%aboveskip=1em,breaklines=true
%}

%
%

%%%%%%%%%%%%%%% 小数点对齐格式 %%%%%%%%%%%%%%%%%%%%%%%%%%%%%%%%%%%%%%%
\newcolumntype{,}{D{.}{,}{5}}  %“,”小数点输入,逗号输出,小数位数最多5位
\newcolumntype{d}[1]{D{.}{.}{#1}} %d{小数位数},输入输出都为.  小数位数自己设定
\newcolumntype{.}{D{.}{.}{-1}} %“.” 输入输出小数点,小数位数为任意值。



%%%%%%%%%%%%%%% 自定义 Verbatim 环境 %%%%%%%%%%%%%%%%%%%%%%%%%%%%%%%%%%%%%%%
\DefineVerbatimEnvironment{VimVBshcmd}{Verbatim}
  {gobble=0,rulecolor=\color{black},formatcom=\color{blue},samepage=true,numbers=none,numbersep=0mm, frame=single,framerule=0.1pt,label=\LaTeX \quad 命令 ,fontsize=\small}
\DefineVerbatimEnvironment{latexcmd}{Verbatim}
  {gobble=0,rulecolor=\color{black},formatcom=\color{blue},samepage=true,numbers=none,numbersep=0mm, frame=single,framerule=0.1pt,label=\LaTeX \quad 命令 ,fontsize=\small}
\DefineVerbatimEnvironment{asycmd}{Verbatim}
  {gobble=0,rulecolor=\color{black},formatcom=\color{blue},samepage=true,numbers=none,numbersep=0mm, frame=single,framerule=0.1pt,label=ASY ~~代码  ,fontsize=\small}
\DefineVerbatimEnvironment{cmd}{Verbatim}
  {gobble=0,rulecolor=\color{black},formatcom=\color{blue},samepage=true,numbers=none,numbersep=0mm, frame=single,framerule=0.1pt,label= 代码命令  ,fontsize=\small}
%参考了 VIM 7.2 手册代码


%%%%%%%%%%%%%%% 自定义分割线%%%%%%%%%%%%%%%%%%%%%%%%%%%%%%%%%%%%%%%
\newcommand{\fengexian}[1]{
{\centering \makebox[\textwidth][s]{
\textcolor{red}{* * * * \mbox{#1} * * * *}}}}


